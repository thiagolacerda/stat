\documentclass[12pt]{article}

\usepackage[utf8]{inputenc}
\inputencoding{utf8}

\title{A Very Simple \LaTeXe{} Template}
\author{
        André Guedes Linhares\\
        Centro de Informática - CIn UFPE\\
            \and
        Thiago de Barros Lacerda\\
        Centro de Informática - CIn UFPE\\
}
\date{\today}

\begin{document}
\maketitle

\section{Introdução}

\paragraph{}
A nova especificação do HTML, o HTML5, inclui um novo elemento visando facilitar o desenvolvimento de aplicações
dinâmicas e fluídas para internet, elemento esse chamado de canvas. De acordo com a especificação do HTML5 [], o canvas
é:
Um bitmap dependente de resolução que pode ser utilizado para renderizar gráficos, elementos gráficos de jogos, arte e
qualquer outro tipo de imagem em tempo real.

O canvas permite ao usuário, adicionar interatividade à paginas web, deixando o usuário controlar gráficos, imagens e
fotos dinamicamente utilizando linguagem de script.

Por causa do canvas, o HTML5 vem sendo vastamente utilizado para desenvolver aplicativos e jogos, não só web, mas também
para dispositivos móveis. Isso se dá ao fato de browsers e engines de renderização HTML darem suporte ao canvas
nativamente, fazendo assim que o HTML5 se torne uma ferramenta muito poderosa para desenvolver aplicativos e jogos que
rodem em diferentes plataformas, usando o mesmo código. Outros pontos importantes que vêm aumentando a utilização do
canvas:
\begin{enumerate}
\item Facilidade de transformar uma página web estática em uma aplicação web dinâmica, para ser usada em smartphones e
tablets.
\item Forte candidato a substituir o Flash, pela sua facilidade de interagir com elementos da página, pois ele não deixa de
ser um elemento HTML.
\item Suportado nativamente nos browsers, o que remove a necessidade de plugins externos.
\end{enumerate}


\section{Objetivos}

\paragraph{}
Este trabalho visa fazer uma avaliação do canvas nos browsers mais utilizados em dispositivos móveis, o Chrome e o
Safari. Para isso utilizamos dois tablets bastante utilizados, o Nexus 7 e o iPad 3.

Nossos experimentos procuraram avaliar em geral qual seria o melhor browser, rodando em qual sistema operacional e
em qual dispositivo, para rodar uma aplicação HTML5 utilizando o canvas.

\section{Metodologia}
Para fazer a avaliação do canvas HTML5 utilizamos o teste em [], que consiste em avaliar operações que são comuns em
jogos e aplicações web utilizando o canvas, tais como: desenho de linhas horizontais e verticais, desenho de múltiplos
retângulos preenchidos, desenho de arcos, desenho de figuras, rotação de objetos, entre outros. O teste tenta fazer
o máximo de operações em um segundo, onde cada operação tem um peso relativo a sua importância e utilização numa
aplicação. No fim uma nota é atribuída ao browser que está rodando o teste, nota esta dependente de quando operações o
browser conseguiu executar.

Como citado na seção anterior, utilizamos o Nexus 7 (rodando o sistema operaional Android) e o iPad 3 (rodando o sistema
operacional iOS). Para os testes no Nexus 7 utilizamos o Chrome (browser mais utilizado nessa plataforma) rodando em
duas versões diferentes de Android, a 4.4 (KitKat) e a 4.1 (Jelly Bean). No iPad 3, rodamos os testes no browser nativo
do sistema, o Safari, utilizando o iOS 7.

\section{Previous work}\label{previous work}
A much longer \LaTeXe{} example was written by Gil~\cite{Einstein}.

\section{Results}\label{results}
In this section we describe the results.

\section{Conclusions}\label{conclusions}
We worked hard, and achieved very little.

\bibliographystyle{abbrv}
\bibliography{stat}

\end{document}
This is never printed
