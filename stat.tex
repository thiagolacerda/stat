\documentclass[12pt]{article}

\usepackage[utf8]{inputenc}
\inputencoding{utf8}

\title{A Very Simple \LaTeXe{} Template}
\author{
        André Guedes Linhares\\
        Centro de Informática - CIn UFPE\\
            \and
        Thiago de Barros Lacerda\\
        Centro de Informática - CIn UFPE\\
}
\date{\today}

\begin{document}
\maketitle

\section{Introdução}

\paragraph{}
A nova especificação do HTML, o HTML5, inclui um novo elemento visando facilitar o desenvolvimento de aplicações
dinâmicas e fluídas para internet, elemento esse chamado de canvas. De acordo com a especificação do HTML5 [], o canvas
é:
Um bitmap dependente de resolução que pode ser utilizado para renderizar gráficos, elementos gráficos de jogos, arte e
qualquer outro tipo de imagem em tempo real.

Por causa do canvas, o HTML5 vem sendo vastamente utilizado para desenvolver aplicativos e jogos, não só web, mas também
para dispositivos móveis. Isso se dá ao fato de browsers e engines de renderização HTML darem suporte ao canvas
nativamente, fazendo assim que o HTML5 se torne uma ferramenta muito poderosa para desenvolver aplicativos e jogos que
rodem em diferentes plataformas, usando o mesmo código.

\section{Previous work}\label{previous work}
A much longer \LaTeXe{} example was written by Gil~\cite{Einstein}.

\section{Results}\label{results}
In this section we describe the results.

\section{Conclusions}\label{conclusions}
We worked hard, and achieved very little.

\bibliographystyle{abbrv}
\bibliography{stat}

\end{document}
This is never printed
