\documentclass[11pt,a4paper]{article}

\usepackage[utf8]{inputenc}
\inputencoding{utf8}
\usepackage{graphicx}
\usepackage{caption}
\usepackage{subcaption}
\usepackage[brazil]{babel}
\usepackage{amsmath,amssymb}

\title{Avaliação do HTML5 canvas em dispositivos móveis}
\author{
        André Guedes Linhares\\
        Centro de Informática - CIn UFPE\\
            \and
        Thiago de Barros Lacerda\\
        Centro de Informática - CIn UFPE\\
}
\date{\today}

\begin{document}
\maketitle

\section{Introdução}

\paragraph{}
A nova especificação do HTML, o HTML5, inclui um novo elemento visando facilitar o desenvolvimento de aplicações
dinâmicas e fluídas para internet, elemento esse chamado de canvas. De acordo com a especificação do HTML5 [], o canvas
é:
Um bitmap dependente de resolução que pode ser utilizado para renderizar gráficos, elementos gráficos de jogos, arte e
qualquer outro tipo de imagem em tempo real.

O canvas permite ao usuário, adicionar interatividade à paginas web, deixando o usuário controlar gráficos, imagens e
fotos dinamicamente utilizando linguagem de script.

Por causa do canvas, o HTML5 vem sendo vastamente utilizado para desenvolver aplicativos e jogos, não só web, mas também
para dispositivos móveis. Isso se dá ao fato de browsers e engines de renderização HTML darem suporte ao canvas
nativamente, fazendo assim que o HTML5 se torne uma ferramenta muito poderosa para desenvolver aplicativos e jogos que
rodem em diferentes plataformas, usando o mesmo código. Outros pontos importantes que vêm aumentando a utilização do
canvas:
\begin{enumerate}
    \item Facilidade de transformar uma página web estática em uma aplicação web dinâmica, para ser usada em smartphones
    e tablets.
    \item Forte candidato a substituir o Flash, pela sua facilidade de interagir com elementos da página, pois ele não
    deixa de ser um elemento HTML.
    \item Suportado nativamente nos browsers, o que remove a necessidade de plugins externos.
\end{enumerate}


\section{Objetivos}

\paragraph{}
Este trabalho visa fazer uma avaliação do canvas nos browsers mais utilizados em dispositivos móveis, o Chrome e o
Safari. Para isso utilizamos dois tablets bastante utilizados, o Nexus 7 e o iPad 3.

Nossos experimentos procuraram avaliar em geral qual seria o melhor browser, rodando em qual sistema operacional e
em qual dispositivo, para rodar uma aplicação HTML5 utilizando o canvas.

\section{Metodologia}
Para fazer a avaliação do canvas HTML5 utilizamos o teste em [], que consiste em avaliar operações que são comuns em
jogos e aplicações web utilizando o canvas, tais como: desenho de linhas horizontais e verticais, desenho de múltiplos
retângulos preenchidos, desenho de arcos, desenho de figuras, rotação de objetos, entre outros. O teste tenta fazer
o máximo de operações em um segundo, onde cada operação tem um peso relativo a sua importância e utilização numa
aplicação. No fim uma nota é atribuída ao browser que está rodando o teste, nota esta dependente de quando operações o
browser conseguiu executar.

Como citado na seção anterior, utilizamos o Nexus 7 (rodando o sistema operaional Android) e o iPad 3 (rodando o sistema
operacional iOS). Para os testes no Nexus 7 utilizamos o Chrome (browser mais utilizado nessa plataforma) rodando em
duas versões diferentes de Android, a 4.4 (KitKat) e a 4.1 (Jelly Bean). No iPad 3, rodamos os testes no browser nativo
do sistema, o Safari, utilizando o iOS 7.

Para cada par [Sistema Operacional, browser], rodamos duas baterias do teste [], uma com 500 amostras e outra com 200.
Com esses dados em mãos fizemos algumas análises estatísticas tendo como objetivo comparar a performance do canvas de
cada par.

\section{Análise exploratória}\label{analise exploratoria}

Para cada par analisado plotamos seus histogramas de densidade e frequência, a ECDF e realizamos um teste de hipótese
entre pares para verificar se suas performances são iguais ou não.

\subsection{ECDFs}\label{ecdfs}

É possível notar na figura~\ref{ipadecdfs} que o desempenho do Safari, rodando no iOS 7 no iPad 3, na grande maioria das
vezes alcança uma nota entre 0.92 e 0.94 e que seu comportamento continua similar mesmo aumentando o número de amostras
analisadas.

\begin{figure}
    \caption{ECDFs - iPad 3 Safari}
    \label{ipadecdfs}
    \begin{subfigure}{.5\textwidth}
        \caption{500 amostras}
        \centering
        \includegraphics[width=\textwidth]{images/ecdf-ipad-3-ios7-safari-500-amostras-20131119}
        \label{safari500}
    \end{subfigure}
    \begin{subfigure}{.5\textwidth}
        \caption{200 amostras}
        \centering
        \includegraphics[width=\textwidth]{images/ecdf-ipad-3-ios7-safari-200-amostras-20131126}
        \label{safari200}
    \end{subfigure}
\end{figure}

Ao comparar o desempenho do Safari com o do Chrome rodando no Android 4.3, podemos perceber, pela
figura~\ref{nexus43ecdfs} que o Chrome se sai melhor do que o Safari, tendo suas notas variando na maioria entre 0.93 e
0.98.

\begin{figure}
    \caption{ECDFs - Nexus 7, Android 4.3 Chrome}
    \label{nexus43ecdfs}
    \begin{subfigure}{.5\textwidth}
        \caption{500 amostras}
        \centering
        \includegraphics[width=\textwidth]{images/ecdf-n7-a43-chrome-500-amostras-20131119}
        \label{nexus43500}
    \end{subfigure}
    \begin{subfigure}{.5\textwidth}
        \caption{200 amostras}
        \centering
        \includegraphics[width=\textwidth]{images/ecdf-n7-a43-chrome-200-amostras-20131120}
        \label{nexus43200}
    \end{subfigure}
\end{figure}

Na figura~\ref{nexus44ecdfs} nota-se que seu desempenho é superior a todos os outros analisados previamente, tendo suas
notas variando, na maioria das vezes, entre 1.5 e 1.6.

\begin{figure}
    \caption{ECDFs - Nexus 7, Android 4.4 Chrome}
    \label{nexus44ecdfs}
    \begin{subfigure}{.5\textwidth}
        \caption{500 amostras}
        \centering
        \includegraphics[width=\textwidth]{images/ecdf-n7-a44-chrome-500-amostras-20131119}
        \label{nexus44500}
    \end{subfigure}
    \begin{subfigure}{.5\textwidth}
        \caption{200 amostras}
        \centering
        \includegraphics[width=\textwidth]{images/ecdf-n7-a44-chrome-200-amostras-20131120}
        \label{nexus44200}
    \end{subfigure}
\end{figure}


\subsection{Histogramas}\label{histogramas}

Abaixo apresentamos os histogramas dos browsers analisados, podemos observar que eles só confirmam as afirmações feitas
na seção anterior, onde o Chrome rodando no Android 4.4 supera todos os outros. Da análise dos histogramas foi possível
concluir também que nenhuma das amostras conseguiram ser modeladas para uma normal, com o intuito de deixar isso claro
foi plotado também a curva normal que mais se aproximaria do histograma.

\begin{figure}
    \caption{Histogramas - iPad 3 Safari}
    \label{safarihistogramas}
    \begin{subfigure}{.5\textwidth}
        \caption{500 amostras}
        \centering
        \includegraphics[width=\textwidth]{images/hist-freq-ipad-3-ios7-safari-500-amostras-20131119}
        \label{safarihistograma500}
    \end{subfigure}
    \begin{subfigure}{.5\textwidth}
        \caption{200 amostras}
        \centering
        \includegraphics[width=\textwidth]{images/hist-freq-ipad-3-ios7-safari-200-amostras-20131126}
        \label{safarihistograma200}
    \end{subfigure}
\end{figure}

Na figura~\ref{nexus43histogramas}, onde apresentamos os histogramas para o Chrome rodand no Android 4.3, foi possível
observer um comportamento bimodal, onde temos dois picos de notas, bem diferente das outras amostras.

\begin{figure}
    \caption{Histogramas - Nexus 7, Android 4.3 Chrome}
    \label{nexus43histogramas}
    \begin{subfigure}{.5\textwidth}
        \caption{500 amostras}
        \centering
        \includegraphics[width=\textwidth]{images/hist-freq-n7-a43-chrome-500-amostras-20131119}
        \label{nexus43histograma500}
    \end{subfigure}
    \begin{subfigure}{.5\textwidth}
        \caption{200 amostras}
        \centering
        \includegraphics[width=\textwidth]{images/hist-freq-n7-a43-chrome-200-amostras-20131120}
        \label{nexus43histograma200}
    \end{subfigure}
\end{figure}

\begin{figure}
    \caption{Histogramas - Nexus 7, Android 4.4 Chrome}
    \label{nexus44histogramas}
    \begin{subfigure}{.5\textwidth}
        \caption{500 amostras}
        \centering
        \includegraphics[width=\textwidth]{images/hist-freq-n7-a44-chrome-500-amostras-20131119}
        \label{nexus44histograma500}
    \end{subfigure}
    \begin{subfigure}{.5\textwidth}
        \caption{200 amostras}
        \centering
        \includegraphics[width=\textwidth]{images/hist-freq-n7-a44-chrome-200-amostras-20131120}
        \label{nexus44histograma200}
    \end{subfigure}
\end{figure}


\subsection{Teste de hipótese}\label{testedehipotese}

Por último, um teste de hipótese foi feito comparando cada dispositivo analisado. Para cada par, criamos as seguintes
hipóteses:

\begin{align*}
        H_0: \mu_0 = \mu_1 \\
        H_1: \mu_0 \neq \mu_1
\end{align*}

Onde a hipótese nula afirma que os dispositivos comparados tem um desempenho igual em média e a hipótese alternativa
contradiz isso, eles não possuem desempenho igual em média.

Visto que os dados das nossas amostras não podem ser aproximados a uma normal, utilizamos teste não paramétrico de
Wilcoxon com o \(\alpha\) de 0.05, para poder testar as hipóteses propostas. A tabela abaixo resume os resultados dos testes.

\begin{center}
    \begin{tabular}{| l | l | l | l |}
    \hline
     & iPad 3 x Android 4.3 & iPad 3 x Android 4.4 & Android 4.3 x Android 4.4 \\ \hline
    \( H_0: \mu_0 = \mu_1 \) & verdadeira & rejeitada & rejeitada \\ \hline
    \( H_0: \mu_0 \neq \mu_1 \) & rejeitada & verdadeira & verdadeira \\ \hline
    \end{tabular}
\end{center}

\section{Conclusão}\label{conclusao}

Com os experimentos apresentados, podemos concluir que, para uma aplicação web utilizando o canvas, o Chrome rodando no
Android 4.4 foi o que apresentou o melhor compartamento

\bibliographystyle{abbrv}
\bibliography{stat}

\end{document}
This is never printed
